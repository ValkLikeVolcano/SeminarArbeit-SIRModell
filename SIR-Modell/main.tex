%----------------------------------------------------------------------------------------
%	PACKAGES AND OTHER DOCUMENT CONFIGURATIONS
%----------------------------------------------------------------------------------------

\documentclass[12pt]{scrartcl} % Font size

%----------------------------------------------------------------------------------------
%	PACKAGES AND OTHER DOCUMENT CONFIGURATIONS
%----------------------------------------------------------------------------------------

\usepackage{amsmath, amsfonts, amsthm} % Math packages

\usepackage{listings} % Code listings, with syntax highlighting

\usepackage[german]{babel} % german language hyphenation

 \usepackage{setspace}

\usepackage[backend=biber, style=numeric-verb]{biblatex}
\addbibresource{literatur.bib}

\usepackage{graphicx} % Required for inserting images
\graphicspath{{Figures/}{./}} % Specifies where to look for included images (trailing slash required)

\usepackage{booktabs} % Required for better horizontal rules in tables

\numberwithin{equation}{section} % Number equations within sections (i.e. 1.1, 1.2, 2.1, 2.2 instead of 1, 2, 3, 4)
\numberwithin{figure}{section} % Number figures within sections (i.e. 1.1, 1.2, 2.1, 2.2 instead of 1, 2, 3, 4)
\numberwithin{table}{section} % Number tables within sections (i.e. 1.1, 1.2, 2.1, 2.2 instead of 1, 2, 3, 4)

\setlength\parindent{0pt} % Removes all indentation from paragraphs

\usepackage{enumitem} % Required for list customisation
\setlist{noitemsep} % No spacing between list items

\usepackage{subcaption}

\captionsetup[subfigure]{list=false}

\usepackage[subfigure]{tocloft}

\newcounter{lofdepth}
\setcounter{lofdepth}{2}

\cftpagenumbersoff{subfigure}

\renewcaptionname{german}{\figurename}{Abb.}
\BeforeStartingTOC[lof]{\def\autodot{:}}

\usepackage{url}

%----------------------------------------------------------------------------------------
%	DOCUMENT MARGINS
%----------------------------------------------------------------------------------------

\usepackage{geometry} % Required for adjusting page dimensions and margins

\geometry{
	paper=a4paper, % Paper size, change to letterpaper for US letter size
	top=2.5cm, % Top margin
	bottom=3cm, % Bottom margin
	left=2.5cm, % Left margin
	right=2.5cm, % Right margin
	headheight=0.75cm, % Header height
	footskip=1.5cm, % Space from the bottom margin to the baseline of the footer
	headsep=0.75cm, % Space from the top margin to the baseline of the header
	%showframe, % Uncomment to show how the type block is set on the page
}

%----------------------------------------------------------------------------------------
%	FONTS
%----------------------------------------------------------------------------------------

\usepackage[utf8]{inputenc} % Required for inputting international characters
\usepackage[T1]{fontenc} % Use 8-bit encoding
\usepackage{textgreek}

\usepackage{mathptmx}
%\usepackage{fourier} % Use the Adobe Utopia font for the document

%----------------------------------------------------------------------------------------
%	SECTION TITLES
%----------------------------------------------------------------------------------------

\usepackage{sectsty} % Allows customising section commands

\sectionfont{\vspace{6pt}\centering\normalfont\scshape} % \section{} styling
\subsectionfont{\normalfont\bfseries} % \subsection{} styling
\subsubsectionfont{\normalfont\itshape} % \subsubsection{} styling
\paragraphfont{\normalfont\scshape} % \paragraph{} styling

%----------------------------------------------------------------------------------------
%	HEADERS AND FOOTERS
%----------------------------------------------------------------------------------------

\usepackage{scrlayer-scrpage} % Required for customising headers and footers

\ohead*{} % Right header
\ihead*{} % Left header
\chead*{} % Centre header

\ofoot*{} % Right footer
\ifoot*{} % Left footer
\cfoot*{\pagemark} % Centre footer
 % Include the file specifying the document structure and custom commands

%----------------------------------------------------------------------------------------
%	TITLE SECTION
%----------------------------------------------------------------------------------------

\title{	
	\normalfont\normalsize
	\vspace{200pt}
	\textsc{Valentin Heider Gymnasium}\\
	\begin{figure}[h] % [h] forces the figure to be output where it is defined in the code (it suppresses floating)
	\centering
	\includegraphics[width=0.5\columnwidth]{VHGLogo.jpg} 
	\end{figure}
	\vspace{25pt}\\
	
	\rule{\linewidth}{0.5pt}\\
	\vspace{20pt}
	{\huge Das SIR - Modell}\\
	\vspace{12pt}
	\rule{\linewidth}{2pt}\\
	\vspace{20pt}
	{\Large W Seminar Mathematik:}\\
	\vspace{12pt}\\
	{\Large Chaos, Fraktale und andere mathematische Faszinationen}\\
	\vspace{15pt}\\
}

\author{\LARGE Nicolas Martin} % Your name

\date{\normalsize 8.11.2022}

\begin{document}

\pagenumbering{Roman}

\maketitle % Print the title
\newpage

\doublespacing
\tableofcontents
\onehalfspacing
\newpage

%----------------------------------------------------------------------------------------

\section{Einleitung}
\pagenumbering{arabic}

Das \textsl{Covid-19} Virus hat in den letzten Jahren Eindrucksvoll bewiesen, wie schnell sich eine Epidemie ausbreiten und zur Pandemie werden kann,  
Sowie welche verheerende Wirkung ebensolche auf die gesamte Bevölkerung und Wirtschaft der Welt haben. 
Immer wieder teilten Experten in den Medien neue Verhaltens- und Hygieneregelungen mit durch welche die Verbreitung des Virus Vorhergesehen  und Verlangsamt werden konnte. Auch Prognosen über die Zukünftige Entwicklung der Pandemie wurden Veröffentlicht. 
Doch wie genau Berechnen Epidemiologen die Entwicklung einer Epidemie um Vorhersagen zu treffen und die geeignetsten Gegenmaßnamen zur Eindämmung einer solchen zu ermitteln?\\
\\
Diese Literaturarbeit soll zunächst das \textbf{SIR-Modell} mit allen Voraussetzungen Betrachten, Die Einhergehenden Differenzialgleichungen und die Basisreproduktionszahl Mathematisch Analysieren und unter Betrachtung der Flexibilität des Modells evaluieren. 
Des weiteren veranschaulicht die Arbeit das Mathematische Chaos des Modells 
ferner welche Veränderungen mit verschiedenen Änderungen der Formeln und Variablen Einhergehen. 
Die Präzision und Effektivität in der Anwendung soll durch Den Fall \textsl{Covid-19} verdeutlicht und mit  HIER FALL EINFÜGEN verglichen werden. 
Ausgenommen Der Fallbeispiele und der Dynamischen Gesamtbevölkerung werden keine Abweichungen des Modells im Detail Besprochen. 

%----------------------------------------------------------------------------------------

\newpage
\section{Das SIR-Modell}

Wenn man den Verlauf einer Epidemie von Anfang bis Ende anhand eines Mathematischen Modells beschreiben möchte, ist das \textbf{SIR-Modell} 
mit nur zwei Parametern und drei Differentialgleichungen in der einfachsten Form ideal, um sich einen groben Überblick zu verschaffen. \cite[vgl.]{4}
Im diesem Kapitel wird auf den Hintergrund, die Möglichkeiten und Limitierungen und die damit Einhergehenden Möglichkeiten der Anwendung des Modells eingegangen.

\subsection{Entstehung}

Das Kermack-McKendrick Modell ist eine Klassische Ausführung des \textbf{SIR-Modells} von den gleichnamigen Infektionsepidemiologen OPEN
why used => einfach anpassbar

%------------------------------------------------

\subsection{Voraussetzungen}

Die erste Bedingung für das SIR-Modell ist die Aufteilung der Gesamtbevölkerung \textit{N} In drei Kategorien:

\begin{itemize}
	\item \textit{S}: Suszeptibel (engl. Susceptible)
	\item \textit{I}: Infiziert (engl. Infected)
	\item \textit{R}: Entfernt (engl. Removed)\\
	\small \textsl{Keine Eindeutigen Fachtermini vorhanden}
\end{itemize}
\normalsize

Wobei (\textit{S}) alle Infizierbaren Individuen, (\textit{I}) alle Infizierten und gleichzeitig Infektiösen und (\textit{R}) alle Genesenen oder verstorbenen beinhaltet.  Zu beachten ist das sich Individuen jeweils immer nur in Einer Gruppe befinden können und sich Individuen der Gruppe 
(\textit{R}) kein weiteres mal Infizieren können. \cite[vgl.]{4}
Das Modell wird von nur zwei Parametern alpha (\textalpha) und beta (\textbeta) Beeinflusst. 
\textalpha \space steht hier für die Übertragungsrate (engl. transmission rate) welche angibt, wie schnell Individuen der Suszeptiblen Gruppe in die Infizierte Gruppe übergehen. 
\textbeta \space hingegen stellt hier die Erholungsrate (engl. recovery rate) dar, und gibt an wie schnell die Gruppe \textit{I} sich erholt oder verstirbt.

Es wird davon ausgegangen das \textit{N}, \textalpha, \textbeta \space Konstant sind.

%------------------------------------------------

\subsection{Anwendung}

%----------------------------------------------------------------------------------------

\newpage
\section{Mathematischer Aufbau}

\textit{N} kann jetzt zu jeder Zeit (\textit{t}) folgendermaßen definiert werden: \cite{3}


$$ \large\textit{S}_{(t)} + \textit{I}_{(t)} + \textit{R}_{(t)} = \textit{N}\normalsize $$


\subsection{Drei Differentialgleichungen}

%------------------------------------------------

\subsection{Basisreproduktionszahl}

%------------------------------------------------

\subsection{Varianten}

%------------------------------------------------

\subsection{Einflussfaktoren}

%----------------------------------------------------------------------------------------

\section{Fallbeispiele}

\subsection{Covid 19}

%------------------------------------------------

\subsection{Malaria oder Ebola Oder so}

%----------------------------------------------------------------------------------------

\section{Fazit}

%----------------------------------------------------------------------------------------

\newpage
\section{Anhang}

%----------------------------------------------------------------------------------------

\newpage
\setlength{\bibitemsep}{\baselineskip}
\printbibliography[heading=bibintoc]

\end{document}
