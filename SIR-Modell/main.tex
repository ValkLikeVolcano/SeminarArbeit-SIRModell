%----------------------------------------------------------------------------------------
%	PACKAGES AND OTHER DOCUMENT CONFIGURATIONS
%----------------------------------------------------------------------------------------

\documentclass[12pt]{scrartcl} % Font size

%----------------------------------------------------------------------------------------
%	PACKAGES AND OTHER DOCUMENT CONFIGURATIONS
%----------------------------------------------------------------------------------------

\usepackage{amsmath, amsfonts, amsthm} % Math packages

\usepackage{listings} % Code listings, with syntax highlighting

\usepackage[german]{babel} % german language hyphenation

 \usepackage{setspace}

\usepackage[backend=biber, style=numeric-verb]{biblatex}
\addbibresource{literatur.bib}

\usepackage{graphicx} % Required for inserting images
\graphicspath{{Figures/}{./}} % Specifies where to look for included images (trailing slash required)

\usepackage{booktabs} % Required for better horizontal rules in tables

\numberwithin{equation}{section} % Number equations within sections (i.e. 1.1, 1.2, 2.1, 2.2 instead of 1, 2, 3, 4)
\numberwithin{figure}{section} % Number figures within sections (i.e. 1.1, 1.2, 2.1, 2.2 instead of 1, 2, 3, 4)
\numberwithin{table}{section} % Number tables within sections (i.e. 1.1, 1.2, 2.1, 2.2 instead of 1, 2, 3, 4)

\setlength\parindent{0pt} % Removes all indentation from paragraphs

\usepackage{enumitem} % Required for list customisation
\setlist{noitemsep} % No spacing between list items

\usepackage{subcaption}

\captionsetup[subfigure]{list=false}

\usepackage[subfigure]{tocloft}

\newcounter{lofdepth}
\setcounter{lofdepth}{2}

\cftpagenumbersoff{subfigure}

\renewcaptionname{german}{\figurename}{Abb.}
\BeforeStartingTOC[lof]{\def\autodot{:}}

\usepackage{url}

%----------------------------------------------------------------------------------------
%	DOCUMENT MARGINS
%----------------------------------------------------------------------------------------

\usepackage{geometry} % Required for adjusting page dimensions and margins

\geometry{
	paper=a4paper, % Paper size, change to letterpaper for US letter size
	top=2.5cm, % Top margin
	bottom=3cm, % Bottom margin
	left=2.5cm, % Left margin
	right=2.5cm, % Right margin
	headheight=0.75cm, % Header height
	footskip=1.5cm, % Space from the bottom margin to the baseline of the footer
	headsep=0.75cm, % Space from the top margin to the baseline of the header
	%showframe, % Uncomment to show how the type block is set on the page
}

%----------------------------------------------------------------------------------------
%	FONTS
%----------------------------------------------------------------------------------------

\usepackage[utf8]{inputenc} % Required for inputting international characters
\usepackage[T1]{fontenc} % Use 8-bit encoding
\usepackage{textgreek}

\usepackage{mathptmx}
%\usepackage{fourier} % Use the Adobe Utopia font for the document

%----------------------------------------------------------------------------------------
%	SECTION TITLES
%----------------------------------------------------------------------------------------

\usepackage{sectsty} % Allows customising section commands

\sectionfont{\vspace{6pt}\centering\normalfont\scshape} % \section{} styling
\subsectionfont{\normalfont\bfseries} % \subsection{} styling
\subsubsectionfont{\normalfont\itshape} % \subsubsection{} styling
\paragraphfont{\normalfont\scshape} % \paragraph{} styling

%----------------------------------------------------------------------------------------
%	HEADERS AND FOOTERS
%----------------------------------------------------------------------------------------

\usepackage{scrlayer-scrpage} % Required for customising headers and footers

\ohead*{} % Right header
\ihead*{} % Left header
\chead*{} % Centre header

\ofoot*{} % Right footer
\ifoot*{} % Left footer
\cfoot*{\pagemark} % Centre footer
 % Include the file specifying the document structure and custom commands

%----------------------------------------------------------------------------------------
%	TITLE SECTION
%----------------------------------------------------------------------------------------

\title{	
	\normalfont\normalsize
	\vspace{200pt}
	\textsc{Valentin Heider Gymnasium}\\
	\begin{figure}[h] % [h] forces the figure to be output where it is defined in the code (it suppresses floating)
	\centering
	\includegraphics[width=0.5\columnwidth]{VHGLogo.jpg} 
	\end{figure}
	\vspace{25pt}\\
	
	\rule{\linewidth}{0.5pt}\\
	\vspace{20pt}
	{\huge Das SIR - Modell}\\
	\vspace{12pt}
	\rule{\linewidth}{2pt}\\
	\vspace{20pt}
	{\Large W Seminar Mathematik:}\\
	\vspace{12pt}\\
	{\Large Chaos, Fraktale und andere mathematische Faszinationen}\\
	\vspace{15pt}\\
}

\author{\LARGE Nicolas Martin} % Your name

\date{\normalsize 8.11.2022} % Today's date (\today) or a custom date

\begin{document}

\pagenumbering{Roman}

\maketitle % Print the title
\newpage

\doublespacing
\tableofcontents
\onehalfspacing
\newpage

%----------------------------------------------------------------------------------------

\section{Einleitung}
\pagenumbering{arabic}

\begin{figure}[h] % [h] forces the figure to be output where it is defined in the code (it suppresses floating)
	\centering
	\includegraphics[width=0.5\columnwidth]{swallow.jpg} % Example image
	\caption{European swallow.}
\end{figure}

%----------------------------------------------------------------------------------------

\section{Das SIR-Modell}

\subsection{Entstehung}

bibba \cite[spage 69]{1}

%------------------------------------------------

\subsection{Funktion}

%------------------------------------------------

\subsection{Verlauf}

%----------------------------------------------------------------------------------------

\section{Aufbau}

\subsection{S I R N}

\begin{align} 
	\label{eq:bayes}
	\begin{split}
		P(A|B) = \frac{P(B|A)P(A)}{P(B)}
	\end{split}					
\end{align}

%------------------------------------------------

\subsection{Alpha Beta}

\begin{align} 
	\begin{split}
		(x+y)^3 &= (x+y)^2(x+y)\\
		&=(x^2+2xy+y^2)(x+y)\\
		&=(x^3+2x^2y+xy^2) + (x^2y+2xy^2+y^3)\\
		&=x^3+3x^2y+3xy^2+y^3
	\end{split}					
\end{align}

%------------------------------------------------

\subsection{Varianten}

%------------------------------------------------

\subsection{Einflussfaktoren}

%----------------------------------------------------------------------------------------

\section{Fallbeispiele}

\subsection{Covid 19}

%------------------------------------------------

\subsection{Malaria oder Ebola Oder so}

%----------------------------------------------------------------------------------------

\section{Fazit}

%----------------------------------------------------------------------------------------

\newpage
\section{Anhang}

%----------------------------------------------------------------------------------------

\newpage
\setlength{\bibitemsep}{\baselineskip}
\printbibliography[heading=bibintoc]

\end{document}
